\RequirePackage{fix-cm}
%
%\documentclass{svjour3}                     % onecolumn (standard format)
%\documentclass[smallcondensed]{svjour3}     % onecolumn (ditto)
\documentclass[smallextended]{svjour3}       % onecolumn (second format)
%\documentclass[twocolumn]{svjour3}          % twocolumn
%
\usepackage{springerpaper}
%
% For the Cahier du GERAD.
%
% \renewcommand{\year}{2022}     % If you don't want the current year.
\newcommand{\cahiernumber}{00}  % Insert your Cahier du GERAD number.
%
% \usepackage{mathptmx}      % use Times fonts if available on your TeX system
%
% insert here the call for the packages your document requires
%\usepackage{latexsym}
% etc.
%
% please place your own definitions here and don't use \def but
% \newcommand{}{}
%
% Insert the name of "your journal" with
\journalname{Mathematical Programming Computation}
%
\def\thistitle{Insert Title Here}
\def\authorone{Dominique Orban}
\def\authortwo{Your Name}
% Meta-information for the PDF file generated..
\hypersetup{
  pdftitle={\thistitle},
  pdfauthor={\authorone{} and \authortwo{}},
  pdfsubject={Paper Subject},
  pdfkeywords={keyword1, keyword2, keyword3},  % TODO: Add keywords
}
%
\begin{document}

\title{%
  \thistitle
  \thanks{%
    Grants or other notes about the article that should go on the front page should be placed here. General acknowledgments should be placed at the end of the article.
  }
}
\subtitle{%
  Do you have a subtitle?\\ If so, write it here
}

%\titlerunning{Short form of title}        % if too long for running head

\author{\authorone{} \and \authortwo{}}

%\authorrunning{Short form of author list} % if too long for running head

\institute{%
  \authorone{} \at
  GERAD and Department of Mathematics and Industrial Engineering, Polytechnique Montr\'eal.         
  \email{dominique.orban@gerad.ca}           %  \\
  % \emph{Present address:} of F. Author  %  if needed
  \and
  \authortwo{} \at
  GERAD and Department of Mathematics and Industrial Engineering, Polytechnique Montr\'eal.         
  \email{your.address@gerad.ca}           %  \\
}

\date{Received: date / Accepted: date}
% The correct dates will be entered by the editor

\linenumbers
\pagestyle{myheadings}

\maketitle
\thispagestyle{mytitlepage}

\begin{abstract}
  Insert your abstract here. Include keywords, PACS and mathematical subject classification numbers as needed.
  \smarttodo{write abstract}
  \keywords{First keyword \and Second keyword \and More}
% \PACS{PACS code1 \and PACS code2 \and more}
% \subclass{MSC code1 \and MSC code2 \and more}
\end{abstract}

% Pour le cahier du GERAD.
%\begin{resume}
%
%\end{resume}
%\textbf{Mots cl\'es :}

\section{Introduction}%
\label{intro}

Your \LaTeX\ source will be managed by Git.
Please write \emph{exactly one sentence per line}, as that will make it much easier to visualize diffs when you make changes.

Your co-authors will have to read it.
Therefore, it should be treated exactly as you treat code.
That means:
\begin{enumerate}
  \item use comments where appropriate;
  \item indent text and \LaTeX\ commands inside environments;
  \item use spaces between commands.
\end{enumerate}

\section{Section title}%
\label{sec:1}

We use Natbib, which means that we do not use the \verb|\cite{}| command.
Instead, we only use Natbib commands, such as \verb|\citep{}| and \verb|\citet{}|.
Please refer to the Natbib cheat sheet at \http{merkel.texture.rocks/Latex/natbib.php}.

\verb|\citep{}| produces \citep{wright-orban-2002} while \verb|\citet{}| produces \citet{wright-orban-2002}.

\subsection{Subsection title}%
\label{sec:2}

Don't forget to give each section and subsection a unique label (see \cref{sec:1}).

\begin{lemma}%
  \label{lem:important}
  This is a lemma.
\end{lemma}

\begin{proof}
  The proof is immediate.
  \qed
\end{proof}

Use \verb|\Cref{}| to refer to environments, including lemmas, theorems, definitions, and sections, which causes the name of the environment and its number to be hyperlinked together.

\begin{theorem}%
  \label{thm:important}
  This theorem relies on \Cref{lem:important}.
\end{theorem}

\begin{proof}
  If a proof ends with a displayed equation, it's possible to place the ``end of proof'' symbol at the end of the equation with \verb|\tag*{\qed}|:
  \[
    e = mc^2.
    \tag*{\qed}
  \]
\end{proof}

\begin{corollary}
  This corollary follows directly from \Cref{thm:important}.
\end{corollary}

\paragraph{Paragraph headings}

Use paragraph headings as needed.
\begin{equation}%
  \label{eq:pythagoras}
  a^2+b^2=c^2.
\end{equation}

However, do not use \verb|\Cref{}| to refer to equations as that would result in \Cref{eq:pythagoras}.
Instead, use \verb|\eqref{}| preceded by an unbreakable space: refer to~\eqref{eq:pythagoras}.

% For one-column wide figures use
% \begin{figure}
% % Use the relevant command to insert your figure file.
% % For example, with the graphicx package use
%   \includegraphics{example.eps}
% % figure caption is below the figure
% \caption{Please write your figure caption here}
% \label{fig:1}       % Give a unique label
% \end{figure}
% %
% % For two-column wide figures use
% \begin{figure*}
% % Use the relevant command to insert your figure file.
% % For example, with the graphicx package use
%   \includegraphics[width=0.75\textwidth]{example.eps}
% % figure caption is below the figure
% \caption{Please write your figure caption here}
% \label{fig:2}       % Give a unique label
% \end{figure*}
%
% For tables use
\begin{table}
  % table caption is above the table
  \caption{Please write your table caption here}
  \label{tab:1}       % Give a unique label
  % For LaTeX tables use
  \begin{tabular}{lll}
    \hline\noalign{\smallskip}
    first & second & third  \\
    \noalign{\smallskip}\hline\noalign{\smallskip}
    number & number & number \\
    number & number & number \\
    \noalign{\smallskip}\hline
  \end{tabular}
\end{table}

%\begin{acknowledgements}
%If you'd like to thank anyone, place your comments here
%and remove the percent signs.
%\end{acknowledgements}

%% References
%% We use Natbib, which is accepted by SIAM and Springer journals
%% see the cheat sheet: http://merkel.texture.rocks/Latex/natbib.php

% Do not use \cite{}. Only use the Natbib commands.
% \citet{jon90}                   -->     Jones et al. (1990)
% \citet[chap. 2]{jon90}          -->     Jones et al. (1990, chap. 2)
% \citep{jon90}                   -->     (Jones et al., 1990)
% \citep[chap. 2]{jon90}          -->     (Jones et al., 1990, chap. 2)
% \citep[see][]{jon90}            -->     (see Jones et al., 1990)
% \citep[see][chap. 2]{jon90}     -->     (see Jones et al., 1990, chap. 2)
% \citet*{jon90}                  -->     Jones, Baker, and Williams (1990)
% \citep*{jon90}                  -->     (Jones, Baker, and Williams, 1990)

%% References with BibTeX database:
% - each reference should have a DOI
% - use the strings provided for the journal name

\bibliographystyle{abbrvnat}
\bibliography{abbrv,\jobname}

\newpage

\hypertarget{contents}{}  % so clicking on [toc] in the header leads here
\tableofcontents
\listoftodos

\end{document}
% end of file template.tex

